\chapter{Preprocessing and Statistics of the Datasets} \label{chapPreproc}

Ako hlavný spôsob riešenia problému generovania prirodzeného textu zo\linebreak[4]štruktúrovaných dát volíme RNN. RNN sú uspôsobené na spracúvanie \linebreak[4]sekvenčných, 1D dát, avšak my potrebujeme spracovať 2D tabuľky. V tejto kapitole popíšem spôsob, ako sa s týmto problémom vyrovnávam, pričom sa pokúsim vysvetliť, prečo je postup rozdielny pre dataset WikiBIO a RotoWire. Ďalej budem rozprávať o tom, ako som ďalej upravoval vstupné a výstupné dáta a dôvody pre každú z aplikovaných zmien.

\section{Transforming Tables to Records}

Najprv sa pokúsim ukázať aký cieľ chcem naplniť pri transformácii tabuľky na sekvenčný vstup. Zadefinujme si tabuľku, s ktorou budeme pracovať. Povedzme, že stĺpce budú označovať typy hodnôt a riadky budú označovať entity. (ako v príklade \ref{ex_struct}). Chceme, aby čo najviac informácií ostalo v dátach. Konkrétne to znamená, že v sekvenčnom vstupe by malo ostať zachované, jednak ku ktorej entite daný vstup patrí, jednak aký typ hodnoty prislúchajúci k danej entite vyjadruje.

\begin{table}
    \centering
    \begin{tabular}{llll}
        \toprule
        {} & field$_1$ & field$_2$ \dots \\
        \midrule
        entity$_1$ & value$_{1,1}$ &  value$_{1,2}$ \dots \\
        entity$_2$ & value$_{2,1}$ & value$_{2,2}$ \dots
    \end{tabular}
    \caption{An example of structured data} \label{ex_struct}
\end{table}

\subsection{Notation}

Podľa \citep{liang-etal-2009-learning} zavádzam notáciu, ktorú budem ďalej používať.\linebreak[4]Tabuľku $\mathcal{T}$ transformujeme na postupnosť záznamov $ \mathbf{s} = \{ r_i \}_{i=1}^{J} $, kde $r_i$ označuje i-ty záznam. Vzhľadom na ciele stanovené vyššie, každý záznam položku $r.f$ označujúcu typ hodnoty, položku $r.v$ označujúcu hodnotu daného záznamu,\linebreak[4]prípadne položku $r.e$, označujúcu entitu, ktorej prislúcha.

\section{WikiBIO}

Transformácie a štatistiky na datasete sa týkajú jednak štruktúrovaných dát, jednak sumárov, ktoré má generačný systém generovať. Štruktúrované dáta majú formu infoboxu, ako sumár slúži prvá veta z príslušného článku na wikipedii. Ako v príklade \ref{masaryk_infobox}. Najprv predstavím štatistiky, následne na ich základe uvediem, aké transformácie a preprocessing som zvolil.

\subsection{Dataset statistics}

Pre tréning neurónovej siete používam origiálny train-valid-test split od autorov, teda 582~659 trénovacích infobox-sumár párov, 72~831 validačných a 72~831 testovacích.

Sumáre obsahujú celkovo 493~878 unikátnych tokenov, celkovo 18~981~222 tokenov. Tabuľky obsahujú 7~200 unikátnych typov. 

\subsection{Transformation of Infoboxes}

Každý infobox sa týka práve jednej entity. To veľmi uľahčuje transformáciu na záznamy, kedže na tabuľku sa možno pozerať ako na zoznam dvojíc\linebreak[4]$(type, \{value_i\}_{i=1}^{|\mathbf{value}|})$. Ako už vyplýva z notácie, počet hodnôt prislúchajúcich jednomu typu môže byť rôznorodý.

\tikzstyle{defStyleWikiBIOEx} = [rectangle, rounded corners, minimum width=3cm, minimum height=1cm,text centered, fill=yellow!10, align=left]

Existujú minimálne dve možnosti ako sa s tým vysporiadať. Prvou je vyhlásiť hodnotu prislúchajúcu jednomu typu za jeden token (ako v príklade \ref{onetokenref}) a\linebreak[4]prípadne vytvoriť viacero typov lepšie reprezentujúcich to, čo sa v tabuľke\linebreak[4]nachádza. To je však vzhľadom na veľkosť datasetu (okolo 730 000 párov infobox-sumár, viac ako 7000 rôznych typov) nevhodný prístup.

\begin{figure}[!h]
\centering
\usetikzlibrary{shapes.multipart}
\begin{tikzpicture}
\node (originalfv1) [defStyleWikiBIOEx] {
    (successor, edvard beneš)
};
\end{tikzpicture}
\caption{An example of a record made from infobox in figure \ref{masaryk_infobox} by treating all the values as one token} \label{onetokenref}
\end{figure}

Druhou možnosťou je prístup, ktorý zvolili aj tvorcovia datasetu \citep{lebret2016neural}, či autori state of the art riešenia \citep{liu2017tabletotext}. Každú hodnotu z množiny hodnôt považujeme za samostatný token. Tabuľka však nemá stálu štruktúru, preto je napríklad problém rozlíšiť, či záznam (successor, edvard) vyjadruje krstné meno, alebo priezvisko nástupcu Tomáša Garrigue Masaryka v jeho funkcii. Preto ku každému záznamu pridávame hodnotu $r.pos$, vyjadrujúcu poradie (číslované od 1) vrámci hodnôt prislúchajúcich jednomu typu, ako je ukázané v príklade \ref{multitokenref}.

\begin{figure}[!h]
\centering
\usetikzlibrary{shapes.multipart}
\begin{tikzpicture}
\node (transformedfv2) [defStyleWikiBIOEx] {
    (successor, edvard, 1), \\ (successor, beneš, 2)
};
\end{tikzpicture}
\caption{An example of a record made from infobox in figure \ref{masaryk_infobox} by adding positional information and treating each value as a separate token} \label{multitokenref}
\end{figure}

 Zatiaľ čo nástupcom T.G. Masaryka bol Edvard Beneš, nástupcom amerického prezidenta Herberta Hoovera bol Franklin D. Roosevelt. Rozpoznať, že obidva záznamy (successor, beneš, 2) a (successor, roosevelt, 3) vyjadrujú priezvisko danej osoby môže byť ťažké. Preto pridávame aj informáciu o poradí od konca, $r.rpos$. Potom je na vybranom príklade už zrejmé, že záznamy s $r.f = successor$ a $r.rpos = 1$ vyjadrujú tú istú informáciu.

\begin{figure}[!h]
\centering
\usetikzlibrary{shapes.multipart}
\begin{tikzpicture}
\node (transformedfv2) [defStyleWikiBIOEx, text width = 0.95*\columnwidth] {
    (name, tomáš, 1, 3), (name, garrigue, 2, 2), (name, masaryk, 3, 1), (image, tomáš, 1, 16), (image, garrigue, 2, 15), (image, masaryk, 3, 14), (image, ",", 4, 13),(image, bain, 5, 12) \dots
};
\end{tikzpicture}
\caption{An example of records made from infobox in figure \ref{masaryk_infobox} with all the additional information included} \label{multitokenfinal}
\end{figure}

\subsection{Preprocessing}

Jednotlivé sety (train, valid, test) ešte prefiltrujem tak, aby žiadna tabuľka nebola dlhšia ako 100 recordov a žiadny sumár nebol dlhší ako 75 tokenov. (urobené na základe štatistík v tabuľke \emph{tu bude odkaz na tabuľku}) Keďže tento dataset nie je nosným datasetom tejto práce, rozhodol som sa neexperimentovať s preprocessingom a použil som hodnoty hyperparametrov, ktoré zvolili \citep{liu2017tabletotext}. V krátkosti zhrniem, čo konkrétne používam.

\subsubsection{General}

Celý dataset je lowercaseovaný, čiarky, zátvorky, bodky \dots sú považované za samostatné tokeny. Všetky okrem najčastejších 20 000 tokenov vybraných \citep{liu2017tabletotext} v sumároch a hodnotách tabuliek nahradím špeciálnymi UNK tokenmi.

\subsubsection{Tables}

Všetky záznamy, kde by hodnota činila \emph{none}, resp. nevalidné hodnoty v tabuľke (prázdne, alebo s nesprávne zadefinovaným typom) odstránim. Taktiež všetky typy, ktoré sa nevyskytujú v slovníku typov od \citep{liu2017tabletotext} (ktorý je zozbieraný zo všetkých typov vyskytujúcich sa aspoň 100-krát) nahradím UNK tokenmi.

\begin{figure}[!h]
\centering
\begin{subfigure}[t]{0.45\textwidth}
    \centering
    \begin{tabular}{ll}
        \toprule
        Records$_1$ & Percentile$_2$\\
        \midrule
        50     & 56,56 \\
        75     & 82,6 \\
        100    & 93,4
    \end{tabular}
\caption{Lenght statistics of tables}
\footnotesize \textit{Note:} $_1$ Number of records, where (f, v, N, M) and (f, v, N+1, M-1) are treated as 2 distinct records \\ $_2$ Percentage of tables with lower number of records
\end{subfigure}
\begin{subfigure}[t]{0.45\textwidth}
    \centering
    \begin{tabular}{ll}
        \toprule
        Tokens$_1$ & Percentile$_2$\\
        \midrule
        25     & 55,76 \\
        50     & 96,84 \\
        75     & 99,68
    \end{tabular}
    \caption{Length statistics of  summaries}
    \footnotesize \textit{Note:} $_1$ Number of tokens in a summary \\ $_2$ Percentage of summaries with lower number of tokens
\end{subfigure}
\caption{Statistics of the WikiBIO dataset} \label{stats_filtering_wb}
\end{figure}

\section{RotoWire}

Na rozdiel od datasetu WikiBIO, v tabuľke charakterizujúcej jeden zápas NBA (teda vo vstupe pre generačný model) sa vyskytuje viacero entít a dokonca viacero druhov entít (tímy, hráči). Tieto fakty je potrebné zobrať do úvahy pri spracúvaní tabuliek a ich transformácii na sekvenčný vstup pre RNN.

\subsection{Dataset Statistics}

Keďže transformácie sumárov sú výrazne ovplyvnené štatistikami datasetu, najprv si dovolím predstaviť to, ako vyzerajú tieto štatistiky. 

\subsection{Transformations of Input Tables}

Ako je spomenuté v sekcii \ref{structured_data_rotowire}, hodnoty v tabuľke môžu byť buď mená tímov, hráčov, miest, alebo celé číslo oznamujúce percentuálnu, či absolútnu hodnotu nejakej štatistiky.

Preto má zmysel použiť rozdielny prístup ku spracovaniu hodnôt ako pri datasete WikiBIO. Konkrétne budeme každú hodnotu považovať za jeden token. Väčšina hodnôt je číselná (konkrétne $\frac{13}{15}$ z typov charakterizujúcich tímy a $\frac{19}{25}$ z typov charakterizujúcich hráčov. Následne mená tímov sú väčšinou dlhé práve jeden token, až na jednu výnimku, z ktorej sa vyrobí jeden token ( \emph{Trail Blazers $\rightarrow$ Trail\_Blazers}). Podobne pre mená tímov a výnimky ( \emph{Oklahoma City $\rightarrow$ Oklahoma\_City}, \emph{Golden State $\rightarrow$ Golden\_State}, \dots).

\subsubsection{Transformations of Player Names} \label{trans_p_nms}

Mená hráčov sú všetky, až na jednu výnimku (\emph{Nene}) viac-tokenové a autori, ktorých prístup ku danému datasetu som bral za referenciu (\citep{wiseman2017}, \citep{puduppully2019datatotext}) zvolili odlišný prístup ako ten, ktorý používam ja.

Referenčný prístup používa 2 špeciálne typy, \emph{first\_name} a \emph{last\_name}. Vďaka nim sa úloha pochopiť, či token \emph{James} odkazuje na krstné meno hviezdneho \emph{James Harden}, alebo na priezvisko legendy \emph{LeBron James} necháva na generačný systém. (Tu by malo význam pridať porovnanie toho, ako to vyzerá u mňa a u nich)

Môj prístup je založený na myšlienke, že už na tak veľmi ťažkom datasete je dôležité vytvoriť čo najjednoduchšiu úlohu pre neurónovú sieť. Preto transformujem sumáre tak, aby meno každého hráča bolo reprezentované práve jedným tokenom. Práve preto strácajú typy \emph{first\_name} a \emph{last\_name} zmysel a vo vstupnej tabuľke ich nemám.

\subsubsection{Entities}

Keďže vstupná tabuľka obsahuje informácie o obidvoch hrajúcich tímoch a\linebreak[4]všetkých hráčoch na súpiskách, každý záznam musí obsahovať aj hodnotu~$r.e$ popisujúcu, ktorú entitu záznam charakterizuje. Vďaka transformácii mien\linebreak[4]hráčov a tímov, môžeme ako entitu použiť názov tímu, resp. meno hráča. Okrem toho však ku každému záznamu pridáme špeciálnu položku $r.ha$, ktorá symbolizuje, či sa záznam vzťahuje ku domácemu, alebo hosťujúcemu tímu. 

\subsubsection{Record Format}

Nakoniec teda používam záznamy, ktoré obsahujú položky $r.f$ - typ záznamu, $r.e$ - entita, $r.v$ - hodnota a $r.ha$ - domáci/hostia. (tu by mal byť pridaný príklad, ako to vyzerá)

\subsection{Transformations of Summaries}

Pri transformácii sumárov vychádzame z dvoch predpokladov:
\begin{itemize}
\item Predpokladáme, že neurónová sieť sa môže naučiť generovať určitý token, pokiaľ sa v tréningových dátach aspoň päťkrát.
\item Predpokladáme, že je jednoduchšou úlohou kopírovať dáta z tabuľky, ako generovať zo skrytého stavu.
\end{itemize}

Kvôli tomu vykonávame také transformácie, aby sa jednak vyskytovalo čo najviac tokenov čo najčastejšie, a aby v sumároch boli používané rovnaké tokeny ako v tabuľke.

\subsubsection{Number Transformations}

Rovnako ako \citep{wiseman2017} a \citep{puduppully2019datatotext}, reprezentujeme čísla len pomocou číslic. Táto transformácia je motivovaná druhým predpokladom a teda tým, že dúfame, že väčšinu čísel bude neurónová sieť kopírovať a nie generovať. To znamená, že napríklad token $five$ transformujeme na $5$. Používame na to knižnicu text2num \footnote[1]{\url{https://github.com/allo-media/text2num}}. Niektoré čísla však majú v basketbalovej terminológii špeciálny význam. Špecificky teda netransformujeme číslo $three$, pretože sa často vyskytuje ako súčasť slovných spojení \emph{three pointer, three pt range \dots}

\subsubsection{Player name transformations}

Ako už vyplýva z \ref{trans_p_nms} a z predpokladov, ktoré sme si stanovili, chceme v texte reprezentovať jednoho hráča práve jedným tokenom. To platí aj pre extrémne prípady ako \emph{Luc Richard Mbah a Moute}, ktorý bol v datasete reprezentovaný šiestimi rôznymi kombináciami elíps v jeho mene. Najčastejšie však vychádza, že v texte sa najprv hráč spomenie a následne sa už oslovuje len priezviskom a až na 17 prípadov (čo činí 2.4\% zo 668 hráčov spomenutých aspoň v jednej tabuľke) je meno hráča vždy zložené z 2 tokenov. (jeden prípad \emph{Nene} je jednotokenový, ostatných 16 je 3 a viac tokenových). (tu pôjde algoritmus, ako vyhľadávam meno hráča)

\subsection{Byte Pair Encoding}

Tu by som chcel predstaviť, o čo ide, ako vyzerá algoritmus, koho implementáciu používam, a ako mi to pomôže v úlohách, ktoré som si stanovil.