\chapter{Experiments} \label{experiments_chapter}
Firstly I would like to elaborate on the selection of the hyperparameters. Then I present few of the generated biographies (based on the input infoboxes from WikiBIO dataset) and some generated summaries (based on the input tables from RotoWire dataset).

\section{Development Methods}

At first I want to talk about the methods taken to find the best hyperparameter settings during the development of models.

To avoid the famous underfitting and overfitting problems (more in section 5.2 in \citep{Goodfellow-et-al-2016}), I use the validation (development) part of the respective dataset 
\section{BLEU}
What it is, why do I use such a metric for evaluating the data.

\section{Manual evaluation}
How the summaries for manual evaluation are chosen, how many people do evaluate the predicted summaries.

\section{Other evaluation approaches}
Which approaches are presented in the read papers, which improvements should be made.

\section{Results of the baseline model}
Learned which teams play, what are the greatest stars of each team, although the summaries diverge, only first few sentences from the generated summaries are relevant.

\section{Dropout}
What it is, where I apply the dropout - on the decoder LSTM cells.

\section{Scheduled Sampling}
What it is, how it solves the divergence of the summaries.

\section{Copy methods}
How do they help the model to choose more relevant data from the table, how do they fare in the concurrence of the baseline model.