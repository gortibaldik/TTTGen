\chapter{Data} \label{chapter:data}

The goal of the thesis is to explore the possibilities of generating natural language representation of a sports match. Several questions arise from the assignment. Are we trying to create a general solution which would work for any sport ? What is the format of the input data ? How should the generated text look like ? In this section I try to answer these questions and present the challenge which we will deal with throughout the thesis.

\section{Narrowing the Task}
The differences in the number of players, the environment, the rules etc. prevent us from creating one system for all the sports, therefore the task will be narrowed to one particular sport, the basketball.

During a match of basketball, a set of statistics about teams and players is gathered. We want to generate a text, which will summarize and pick the most important information from the statistics.

\section{Data requirements} \label{data_requirements_section}

Our approach is to train a deep neural network model. We opt for supervized training regime where each input table is associated with a target text, and the model learns a conditional probability $p(text | input\_table)$. 

In the introduction \ref{IntroductionChapter} I already pointed out that neural networks are excellent at learning a language model. However we want to learn to generate a text conditioned on the \emph{concrete match statistics}.

This places high demands on the quality of the texts which are used as targets during training. The summaries should be connected to statistical data about the match, and contain the least possible amount of subjective, emotional information. Otherwise the model wouln't have possibility to learn how to interpret its inputs, and instead it would learn how to generate some invaluable emotional phrases.

It is known that a lot of data is needed to train a deep neural network (e.g. \citep{sennrich2016} trains the neural machine translation system on 4.2 million English-German sequence pairs). Therefore it's unfeasible to collect and prepare our own dataset, and we choose to select from a high variety of publicly available datasets.

I've chosen the RotoWire dataset \citep{wiseman2017}. It contains a set of statistics-summary pairs, summaries being extracted from the portal focused on fantasy sport news, \url{https://www.rotowire.com/basketball/}.

\section{Summaries}

In section \ref{data_requirements_section} we have already discussed our demands on the summaries which will be used as targets.

\citep{wiseman2017} experimented with one other summary origin, \url{https://www.sbnation.com/nba}, where articles are written by fans for fans. However such summaries haven't met the requirements ("many documents in the dataset focus on information not in the box- and line-scores") and the neural networks trained on these kind of summaries "performed poorly".

Therefore we experiment only with the summaries from Rotowire fantasy sports news.

\subsection{Fantasy sports}

According to \citep{Tozzi1999}, the origins of the phenomenon of fantasy sports can be dated to the beginning of 1960s. The article tells a legend about a restaurant called La Rotisserie Francaise. There a group of "sportswriters, accountants and lawyers" started the first fantasy sport league. Rotowire, which name is derived from the name of the restaurant, is the source of all the target summaries in the dataset.

Let's proceed and discuss the rules of the fantasy sport according to Rotowire\footnote{\url{https://www.rotowire.com/basketball/advice/} \label{footnote_2_fs}}. The fantasy league is created on top of a real world league, e.g. NBA. If you want to play, you choose a subset of NBA players and gain points based on their performance in the real-world NBA matches. You have limited resources and better players cost more. The selection must contain a basketballer for each position (therefore it is not possible to draft e.g. 5 point guards and no power forward). The points are awarded according to the role, so that defensive and offensive players gain you points for different achievements in a match. The leagues differ in their specifics and I advise an interested reader to check the Rotowire fantasy league rules\textsuperscript{\ref{footnote_2_fs}}.

\subsection{Fantasy sports news}

To succeed in the fantasy sport, one must keep a good track of the player statistics and injuries, the tendencies of team performances, the trends causing the emergence of a future star or a burn-out. From the beginning there exists a form of news specializing directly on fantasy league players. \url{https://www.rotowire.com/} is one of the most known ones. The articles tend to summarize the most important statistics as well as present a deeper understanding of the events that happened during the match. As noted by \citep{wiseman2017}, the articles are the ideal candidates for the target summaries of the structured data from the dataset.

\section{Structured Data} \label{structured_data_rotowire}

The structured data is in form of multiple tables of statistics related to a particular NBA match. In this section I examine the tables, and hypothesize about their possible roles in the generation of a summary.

To begin the examination, each datapoint contains a date when the match was played. This information isn't leveraged in the summaries\footnote{This isn't technically true, as the majority of summaries contain a phrase like \emph{"Team A defeated Team B on monday."}, however since the date is in form \emph{DD/MM/YYYY}, there isn't any chance the network could deduce the day of the week from the date.}, therefore I opt not to use it.

\subsection{Team Statistics}

A datapoint contains a separate table with statistics, called the \emph{line score}, for each of the opposing teams. Each \emph{line score} contains 15 fields, some of which we will discuss in this section. The full listing can be found on the official github of the dataset \footnote{\label{note1}\url{https://github.com/harvardnlp/boxscore-data}}

Firstly there are fields with the team name and the city where the team is located.

Next there is a set of numerical statistics. We can divide them to the contextual statistics (the number of wins and looses prior to the actual match), the team-totals (\emph{Team-total} means the overall team statistics. E.g. \emph{TEAM-PTS} is the sum of all the points scored by the players of the team), and the per-period point statistics (\emph{TEAM-PTS\_QTR1} - \emph{TEAM-PTS\_QTR4}). Table \ref{tab_team_stats} is a part of both line scores taken from the Rotowire dataset.

\begin{table}[bh!]
    \centering
    \begin{tabular}{lllllllll}
        \toprule
        Name    & City         & PTS$_1$ & AST$_2$ & REB$_3$ & TOV$_4$ & Wins & Losses  &\dots \\
        \midrule
        Raptors & Toronto      & 122 & 22  & 42  & 12  & 11   & 6       &\dots \\
        76ers   & Philadelphia & 95  & 27  & 38  & 14  & 4    & 14     &\dots \\
        \bottomrule
        \multicolumn{9}{l}{\footnotesize \textit{Note:} The statistics are accumulated across all the team players} \\
        \multicolumn{9}{l}{\footnotesize $_1$ Points; $_2$ Assists; $_3$ Rebounds; $_4$ Turnovers}    \end{tabular}
    \caption{\centering An example of the team statistics from the development part of the Rotowire dataset} \label{tab_team_stats}
\end{table}

\subsection{Player statistics}

The player statistics are gathered in a table called \emph{box score}. In each row there is an information about the player name, the city he plays for (since there are 2 teams originating in Los Angeles, the distinction is made by calling one \emph{LA} and the other one \emph{"Los Angeles"}), his position on the starting roster (or the \emph{N/A} value if he isn't in the starting lineup), and another 19 fields with statistics summing up the impact of the player in the match (e.g. points, assists, minutes played etc.). Again, the full listing can be found on github\textsuperscript{\ref{note1}} and table \ref{tab_player_stats} contains an example of a \emph{box score} from the dataset.

\begin{table}[h!]
    \begin{tabular}{lllllll}
        \toprule
        Name             & Team City    & S\_POS$_1$ & PTS$_2$ & STL$_3$ & BLK$_4$        \dots \\
        \midrule
        Kyle Lowry       & Toronto      & G         & 24  & 1   & 0       \dots \\
        Terrence Ross    & Toronto      & N/A       & 22  & 0   & 0        \dots \\
        Robert Covington & Philadelphia & G         & 20  & 2   & 0        \dots \\
        Jahlil Okafor    & Philadelphia & C         & 15  & 0   & 1        \dots \\
        \bottomrule
        \multicolumn{6}{l}{\footnotesize \textit{Note:} $N/A$ means that the statistic couldn't be collected because it is undefined} \\
        \multicolumn{6}{l}{\footnotesize (e.g. player didn't appear on starting roster therefore his starting position is undefined)} \\
        \multicolumn{6}{l}{\footnotesize $_1$ Starting position ; $_2$ Points; $_3$ Steals; $_4$ Blocks}
    \end{tabular}
    \caption{\centering An example of the player statistics from the development part of the Rotowire dataset}\label{tab_player_stats}
\end{table}

\section{Relation of summaries and tables}
\definecolor{lightblue}{rgb}{.15,.77,.90}

In this section I would like to show the relationship between the structured data and the target summary. Let's observe the data-point from the development part of the dataset in figure \ref{fig:samplesummary}. The non-highlighted text shows one-to-one correspondence with the structured data. The \sethlcolor{lightblue} \hl{blue-highlighted text} marks the information which is present in the input data only implicitly. However the level of implicitness varies. It is relatively easy to see that since Terrence Ross's starting position is "N/A", he must have started off the bench. On the other hand the fact that "The Raptors came into this game as a monster favorite" requires comparison of the winning-loosing records of both teams. The \sethlcolor{yellow} \hl{yellow-highlighted text} labels the information that isn't deducible from the input data and the network would have to learn to hallucinate to be able to generate such text.

Somebody may argue that the observations do not sound favorable for the dataset. I see it as a big challenge and a possibility to apply advanced preprocessing as well as interesting architectural design of the neural networks. I will further elaborate on the issue of too noisy data in section \ref{cleaning_section}.

\definecolor{lightgreen}{rgb}{.7,.9,.1}

\begin{figure*}[h!]
\centering
\scalebox{0.85}{
\begin{tikzpicture}


\node(tables)[draw, inner sep=5pt, rounded corners, text width=39em]{
    \small
    \begin{center}
        \begin{tabular}{lcccccc}
        \toprule
        {}      & WIN & LOSS & PTS$_1$ & FG\_PCT$_2$ & REB$_3$ & AST$_4$ \ldots \\
        TEAM    &     &      &     &         &    &           \\
        \midrule
        Raptors & 11  & 6    & 122 & 55      & 42 & 22        \\
        76ers   & 4   & 14   & 95  & 42      & 38 & 27        \\
        \bottomrule
        \end{tabular}
        \vspace{0.5cm}

        \begin{tabular}{llllllll}
            \toprule
            {}                & City         & PTS$_1$ & AST$_4$ & REB$_3$ & FG$_5$  & FGA$_6$ & S\_POS$_7$ $\ldots$ \\
            PLAYER            &              &     &     &     &     &     &                 \\
            \midrule
            Kyle Lowry        & Toronto      & 24  & 8   & 4   & 7   & 9   & G               \\
            Terrence Ross     & Toronto      & 22  & 0   & 3   & 8   & 11  & N/A             \\
            Robert Covington  & Philadelphia & 20  & 2   & 5   & 7   & 11  & G               \\
            Jahlil Okafor     & Philadelphia & 15  & 0   & 5   & 7   & 14  & C               \\
            DeMar DeRozan     & Toronto      & 14  & 5   & 5   & 4   & 13  & G               \\
            Jonas Valanciunas & Toronto      & 12  & 0   & 11  & 6   & 12  & C               \\
            Ersan Ilyasova    & Philadelphia & 11  & 3   & 6   & 4   & 8   & F               \\
            Sergio Rodriguez  & Philadelphia & 11  & 7   & 3   & 4   & 7   & G               \\
            Richaun Holmes    & Philadelphia & 11  & 1   & 9   & 4   & 10  & N/A             \\
            Nik Stauskas      & Philadelphia & 11  & 2   & 0   & 4   & 9   & N/A             \\
            Joel Embiid       & Philadelphia & N/A & N/A & N/A & N/A & N/A & N/A             \\
            \ldots
        \end{tabular}
    \end{center}
}; % end of node
\node(summary) [rectangle, draw,thick,fill=blue!0,text width=39em, rounded corners, inner sep =8pt, minimum height=1em, below=-2mm of tables]{
    \baselineskip=100pt
    \small
    The host Toronto Raptors defeated the Philadelphia 76ers , 122 - 95 , \sethlcolor{yellow} \hl{at Air Canada Center on Monday} . \sethlcolor{lightblue} \hl{The Raptors came into this game as a monster favorite} and they did n't leave any doubt with this result . Toronto just continuously piled it on , as they won each quarter by at least four points . The Raptors were lights - out shooting , as they went 55 percent from the field and 68 percent from three - point range . They also held the Sixers to just 42 percent from the field and dominated the defensive rebounding , 34 - 26 . Fastbreak points was a huge difference as well , with Toronto winning that battle , 21 - 6 . \sethlcolor{lightblue} \hl{Philadelphia ( 4 - 14 ) had to play this game without Joel Embiid }\sethlcolor{yellow} \hl{( rest )} and they clearly did n't have enough to compete with a potent Raptors squad . Robert Covington \sethlcolor{lightblue} \hl{had one of his best games of the season though} , tallying 20 points , five rebounds , two assists and two steals on 7 - of - 11 shooting . Jahlil Okafor \sethlcolor{lightblue} \hl{got the start for Embiid} and finished with 15 points and five rebounds . Sergio Rodriguez , Ersan Ilyasova , Nik Stauskas and Richaun Holmes all finished with 11 points a piece . \sethlcolor{yellow} \hl{The Sixers will return to action on Wednesday , when they host the Sacramento Kings for their next game .} Toronto ( 11 - 6 ) left very little doubt in this game who the more superior team is . Kyle Lowry carried the load for the Raptors , accumulating 24 points , four rebounds and eight assists . \sethlcolor{lightblue} \hl{Terrence Ross was great off the bench , scoring 22 points on 8 - of - 11 shooting .} DeMar DeRozan finished with 14 points , five rebounds and five assists . Jonas Valanciunas recorded a double - double , totaling 12 points and 11 rebounds . \sethlcolor{yellow} \hl{The Raptors next game will be on Wednesday , when they host the defensively - sound Memphis Grizzlies .} 
    \par
};
\node[rectangle, below=2mm of summary, text width=40em] {
    \footnotesize \textit{Note:} $_1$ Points; $_2$ Field Goal Percentage; $_3$ Rebounds; $_4$ Assists; $_5$ Field Goals; $_6$ Field Goals Attempted; $_7$ Starting Position; $N/A$ means undefined value
};
\end{tikzpicture}
}
\caption{ \small A data-point from the development part of the Rotowire dataset. Yellow-highlighted text isn't based on the input data. Blue-highlighted text implicitly follows from the input data.
}
\label{fig:samplesummary}
\end{figure*}
