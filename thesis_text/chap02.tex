\chapter{Data}

One needs a lot of data if he wants to train his neural network. E.g. \citep{sennrich2016} trains the neural machine translation system on 4.2 million English-German sequence pairs. The Data-to-Text generation task má vyššie nároky na kvalitu datasetu. Potrebujeme, aby boli vstupné dáta štandardizované a aby výstupné texty zodpovedali vstupným dátam. Existuje viacero datasetov, ktoré spĺňajú túto podmienku. V tejto kapitole predstavím datasety WikiBIO a Rotowire.

\section{General description}

\showboxdepth=5
\showboxbreadth=5

Obidva tieto datasety používajú notáciu, ktorá bola predstavená v článku od \citep{liang-etal-2009-learning}, preto ju najprv aj tu zadefinujeme.

Ako vstup používame postupnosť záznamov (recordov) $ \mathbf{s} = \{ r_i \}_{i=1}^{J} $. Každý record $ r $ má svoj typ $ r.t \in \mathcal{T} $. Množina typov $\mathcal{T}$ je dopredu definovaná. Ďalej má typ, množinu hodnôt $ r.v = \{ r.v_1, ... , r.f_m\}$. Napríklad v datasete WeatherGOV: $ r.t == windSpeed $, $ r.v = \{time, min, mean, max, mode\}$. Na základe týchto záznamov následne predpovedáme výstupný text $ \mathbf{w} = \{ w_i\}_{i=1}^{ | \mathbf{w} | }$

\section{WikiBIO dataset}

Štruktúrované dáta vo WikiBIO datasete sú vo forme tabuliek. Encoder-Decoder architektúra však vyžaduje, aby do nej boli dáta vkladané sekvenčne. [potrebujem tu pridať príklad tabuľky, pomocou obrázka]. Preto po vzore \citep{lebret2016neural} je tabuľka sploštená do série recordov.Typy sú anotácie riadkov, napríklad meno, dátum narodenia. Príslušných hodnôt však môže byť variabilne veľa. Preto musia vytvorené recordy reflektovať aj poradie hodnôt, čo dosahujeme tým, že pridáme pozičnú informáciu. Recordy sú teda podávané vo formáte $ name_1 = Frantisek, name_2 = Trebuna, birthplace_1 = Kosice, birthplace_2 = Slovensko \dots$ -- pridať informáciu o odlišných dĺžkach tabuliek -- lepší príklad, než vlastné meno --

\subsection{Statistics of WikiBIO dataset}
o tomto píšem zajtra

\subsection {Preprocessing of WikiBIO dataset}
o tomto píšem zajtra


\section{Rotowire dataset}

Pre samotnú úlohu generovania textového popisu zápasu zo štruktúrovaných dát som si vybral dataset RotoWire \citep{wiseman2017}. Štruktúrované dáta sú vo forme tabuľky. Tabuľka obsahuje jednak hodnoty popisujúce celkové tímové štatistiky, jednak hodnoty, popisujúce štatistiky hráčov. Ako výstupné dáta slúžia sumáre jednotlivých zápasov z portálu venujúcemu sa real-time fantasy sports news \url{https://www.rotowire.com/basketball/} (see below).

\subsection{Structured data}

V tejto podsekcii si rozoberieme, aké dáta sa v tabuľke nachádzajú, a čo z nich sa dá využiť pri vytváraní sumáru.

Z vygenerovaného sumáru má byť mimo iné jasné, ktoré 2 tímy hrali a ktorý z nich vyhral. To sa dá zistiť, z mien tímov a počtu bodov, ktoré každý tím strelil. Lepšie porozumenie tabuľky dovolí napríklad zistiť, či vyhral favorit ( teda či tím, ktorý vyhral má v sezóne lepší pomer výhier a prehier ako jeho súper), či bola výhra šťastná ( nezvykle vysoká úspešnosť streľby z poľa a streľby za 3 body), ktorý tím hral lepšie defenzívne (nižší počet turnoverov) atď. Všetky tieto dáta pomenúvame ako tímové štatistiky. Celkovo obsahuje tabuľka tímových štatistík 15 údajov a v datasete sa označuje ako \emph{line score}. Príkladom môže byť tabuľka \ref{tab_team_stats} vytiahnutá z validačného datasetu.

Štruktúrované dáta ale neobsahujú len tímové štatistiky, ale aj hráčske\linebreak[4]štatistiky. Najjednoduchšie využitie hráčskych štatistík je výber hráča, ktorý strelil najviac bodov v danom tíme. Lepšie pochopenie môže dovoliť vypichnúť hráča s veľkým počtom blokov, alebo stealov pokiaľ hral tím defenzívne, alebo hráča s najväčším počtom trojok, pokiaľ mal tím vysokú úspešnosť spoza trojkového oblúka. Tabuľka hráčskych štatistík obsahuje 24 údajov. Príkladom hráčskych štatistík môže byť tabuľka \ref{tab_player_stats} vytiahnutá z validačného datasetu.

Okrem týchto štatistických údajov ešte vstupná tabuľka obsahuje deň, v ktorý sa zápas hral.


\begin{table}[!b]
    \begin{tabular}{llllllll}
        \toprule
        Name    & City         & PTS$_1$ & AST$_2$ & REB$_3$ & TOV$_4$ & Wins & Losses  \dots \\
        \midrule
        Raptors & Toronto      & 122 & 22  & 42  & 12  & 11   & 6       \dots \\
        76ers   & Philadelphia & 95  & 27  & 38  & 14  & 4    & 14      \dots \\
        \bottomrule
        \multicolumn{8}{l}{\footnotesize \textit{Note:} The statistics are accumulated across all the team players} \\
        \multicolumn{8}{l}{\footnotesize $_1$ Points; $_2$ Assists; $_3$ Rebounds; $_4$ Turnovers}
    \end{tabular}
    \caption{Príklad tímových štatistík z datasetu Rotowire}\label{tab_team_stats}
\end{table}

\begin{table}[!b]
    \begin{tabular}{lllllll}
        \toprule
        Name             & Team City    & S\_POS$_1$ & PTS$_2$ & STL$_3$ & BLK$_4$        \dots \\
        \midrule
        Kyle Lowry       & Toronto      & G         & 24  & 1   & 0       \dots \\
        Terrence Ross    & Toronto      & N/A       & 22  & 0   & 0        \dots \\
        Robert Covington & Philadelphia & G         & 20  & 2   & 0        \dots \\
        Jahlil Okafor    & Philadelphia & C         & 15  & 0   & 1        \dots \\
        \bottomrule
        \multicolumn{6}{l}{\footnotesize \textit{Note:} $N/A$ means that the statistic couldn't be collected because it is undefined} \\
        \multicolumn{6}{l}{\footnotesize (e.g. player didn't appear on starting roster therefore his starting position is undefined)} \\
        \multicolumn{6}{l}{\footnotesize $_1$ Starting position ; $_2$ Points; $_3$ Steals; $_4$ Blocks}
    \end{tabular}
    \caption{Príklad hráčskych štatistík z datasetu Rotowire}\label{tab_player_stats}
\end{table}

\subsection{Summaries}

Kladieme vysoké nároky na sumáre. Jednak musia súvisieť s tabuľkou, jednak nechceme neurónovú sieť učiť niečo, čo by sa dalo rýchlo napísať pomocou jednoduchej šablóny. Preto potrebujeme sumáre, ktoré dokážu vytiahnuť z tabuľky hlbšie štatistiky a správne s nimi pracovať. Ako už vysvetľuje \citep{wiseman2017} obyčajné športové spravodajstvo zo stránok \url{https://www.sbnation.com/nba} nespĺňa tieto kritériá, keďže obsahuje priveľa informácií, ktoré sú založené na iných kontextoch, ako sú dáta v zápasovej tabuľke. Práve preto je zaujímavé sledovať fenomén fantasy športov.


\subsubsection{Fantasy sports}

Podľa \citep{Tozzi1999} môžeme datovať počiatky fenoménu fantasy športu do\linebreak[4]šesťdesiatych rokov dvadsiateho storočia. Podľa stránky RotoWire\footnote[1]{\url{https://www.rotowire.com/basketball/advice/}} je základom fantasy športu vytvorenie tímov reálnych hráčov z ligy a získavanie bodov na základe ich výkonov v skutočných zápasoch. Bodovanie berie do úvahy, či ide o defenzívneho, alebo ofenzívneho hráča a je nutné vybrať hráča na každú pozíciu v hre. (čím sa zaručuje, že nemôžu existovať fantasy tímy, ktoré majú len point guardov). Disponujete obmedzenými zdrojmi na nákup hráčov a hráči, u ktorých je pravdepodobnejšie, že skórujú viac bodov, resp. získajú viac lôpt a tak podobne, stoja viac. Skutočné pravidlá sú jemne zložitejšie a tí, ktorých by fantasy ligy zaujímali odporúčame na dané stránky\footnote[1].

\subsubsection{Fantasy sports news}

Na to, aby hráč mohol uspieť vo fantasy lige, potrebuje mať dokonalý prehľad o štatistikách hráčov, o trendoch, o zraneniach, o tímoch, ktorým sa darí aj o tých, ktorým sa možno začne dariť neskôr. Práve preto už od začiatku organizovania fantasy líg existujú formy spravodajstva, ktoré sa špecializujú práve na hráčov fantasy športov. Podľa legendy popísanej \citep{Tozzi1999} sa hráči jednej z prvých fantasy líg schádzali v reštaurácii La Rotisserie Francaise, a podľa tejto reštaurácie je pomenovaná aj stránka špecializujúca sa na spracovávanie štatistík pre fanúšikov fantasy líg, \url{https://www.rotowire.com/}. Články písané o zápasoch oveľa viac berú do úvahy, čo sa v zápase udialo a zároveň poskytujú aj hlbší náhľad do kontextu zápasu. Preto ako píše \citep{wiseman2017} je pre generačné systémy ideálnejšie učiť sa generovať práve články podobné tým z RotoWire. 

\subsection{Relation of summaries and tables}
\definecolor{lightblue}{rgb}{.15,.77,.90}

V tejto kapitole na jednoduchom príklade ukážem, ako sú štatistiky z tabuľky previazané so sumárom z RotoWire. Vo figure \ref{fig:samplesummary} je krásne vidieť, že väčšina údajov zo sumáru pochádza z tabuľky. Napriek tomu však text obsahuje niektoré údaje ( \sethlcolor{yellow} \hl{zvýraznené žltou farbou} ), ktoré v tabuľke vôbec nie sú a niektoré údaje (\sethlcolor{lightblue} \hl{zvýraznené modrou farbou} ), ktoré sú v tabuľke len implicitne a je ich potrebné odvodiť. Zatiaľ čo fakt, že keďže Terrence Ross má nedefinovanú štartovaciu pozíciu, tak musel nastúpiť do zápasu z lavičky je celkom zrejmý, informácia o tom, že Joel Embiid nehrá je dôležitá len vtedy, pokiaľ je Joel Embiid natoľko dôležitý hráč pre tím, že stojí za zmienku ho spomenúť. To je však kontext, ktorý v tabuľke spomenutý nie je a navyše informácia o tom, prečo nehrá nemôže byť jasná ani z kontextu celého korpusu dát.

\definecolor{lightgreen}{rgb}{.7,.9,.1}

\begin{figure*}
\centering
\scalebox{0.85}{
\begin{tikzpicture}


\node(tables)[draw, inner sep=5pt, rounded corners, text width=39em]{
    \small
    \begin{center}
        \begin{tabular}{lcccccc}
        \toprule
        {}      & WIN & LOSS & PTS$_1$ & FG\_PCT$_2$ & REB$_3$ & AST$_4$ \ldots \\
        TEAM    &     &      &     &         &    &           \\
        \midrule
        Raptors & 11  & 6    & 122 & 55      & 42 & 22        \\
        76ers   & 4   & 14   & 95  & 42      & 38 & 27        \\
        \bottomrule
        \end{tabular}
        \vspace{0.5cm}

        \begin{tabular}{llllllll}
            \toprule
            {}                & City         & PTS$_1$ & AST$_4$ & REB$_3$ & FG$_5$  & FGA$_6$ & S\_POS$_1$ $\ldots$ \\
            PLAYER            &              &     &     &     &     &     &                 \\
            \midrule
            Kyle Lowry        & Toronto      & 24  & 8   & 4   & 7   & 9   & G               \\
            Terrence Ross     & Toronto      & 22  & 0   & 3   & 8   & 11  & N/A             \\
            Robert Covington  & Philadelphia & 20  & 2   & 5   & 7   & 11  & G               \\
            Jahlil Okafor     & Philadelphia & 15  & 0   & 5   & 7   & 14  & C               \\
            DeMar DeRozan     & Toronto      & 14  & 5   & 5   & 4   & 13  & G               \\
            Jonas Valanciunas & Toronto      & 12  & 0   & 11  & 6   & 12  & C               \\
            Ersan Ilyasova    & Philadelphia & 11  & 3   & 6   & 4   & 8   & F               \\
            Sergio Rodriguez  & Philadelphia & 11  & 7   & 3   & 4   & 7   & G               \\
            Richaun Holmes    & Philadelphia & 11  & 1   & 9   & 4   & 10  & N/A             \\
            Nik Stauskas      & Philadelphia & 11  & 2   & 0   & 4   & 9   & N/A             \\
            Joel Embiid       & Philadelphia & N/A & N/A & N/A & N/A & N/A & N/A             \\
            \ldots
        \end{tabular}
    \end{center}
}; % end of node
\node(summary) [rectangle, draw,thick,fill=blue!0,text width=39em, rounded corners, inner sep =8pt, minimum height=1em, below=-2mm of tables]{
    \baselineskip=100pt
    \small
    The host Toronto Raptors defeated the Philadelphia 76ers , 122 - 95 , \sethlcolor{yellow} \hl{at Air Canada Center on Monday} . \sethlcolor{lightblue} \hl{The Raptors came into this game as a monster favorite} and they did n't leave any doubt with this result . Toronto just continuously piled it on , as they won each quarter by at least four points . The Raptors were lights - out shooting , as they went 55 percent from the field and 68 percent from three - point range . They also held the Sixers to just 42 percent from the field and dominated the defensive rebounding , 34 - 26 . Fastbreak points was a huge difference as well , with Toronto winning that battle , 21 - 6 . \sethlcolor{lightblue} \hl{Philadelphia ( 4 - 14 ) had to play this game without Joel Embiid }\sethlcolor{yellow} \hl{( rest )} and they clearly did n't have enough to compete with a potent Raptors squad . Robert Covington \sethlcolor{lightblue} \hl{had one of his best games of the season though} , tallying 20 points , five rebounds , two assists and two steals on 7 - of - 11 shooting . Jahlil Okafor \sethlcolor{lightblue} \hl{got the start for Embiid} and finished with 15 points and five rebounds . Sergio Rodriguez , Ersan Ilyasova , Nik Stauskas and Richaun Holmes all finished with 11 points a piece . \sethlcolor{yellow} \hl{The Sixers will return to action on Wednesday , when they host the Sacramento Kings for their next game .} Toronto ( 11 - 6 ) left very little doubt in this game who the more superior team is . Kyle Lowry carried the load for the Raptors , accumulating 24 points , four rebounds and eight assists . \sethlcolor{lightblue} \hl{Terrence Ross was great off the bench , scoring 22 points on 8 - of - 11 shooting .} DeMar DeRozan finished with 14 points , five rebounds and five assists . Jonas Valanciunas recorded a double - double , totaling 12 points and 11 rebounds . \sethlcolor{yellow} \hl{The Raptors next game will be on Wednesday , when they host the defensively - sound Memphis Grizzlies .} 
    \par
};
\node[rectangle, below=2mm of summary, text width=40em] {
    \footnotesize \textit{Note:} $_1$ Points; $_2$ Field Goal Percentage; $_3$ Rebounds; $_4$ Assists; $_5$ Field Goals; $_6$ Field Goals Attempted;
    \footnotesize $N/A$ means undefined value
};
\end{tikzpicture}
}
\caption{ \small Príklad vstupných tabuliek a zlatého sumáru z datasetu. Žltou sú zvýraznené informácie nenachádzajúce sa v tabuľke, zatiaľ čo modrou sú zvýraznené informácie, ktoré z tabuľky a celého korpusu vyplývajú len implicitne.
}
\label{fig:samplesummary}
\end{figure*}


\subsection{The statistics of the dataset}
General statistics of the dataset before preprocessing - number of tokens, unique tokens, player names, city names. Why we want to make the number of tokens lower while keeping the ability to have rich vocabulary.

\subsection{Cleaning}
Lorem Ipsum as one of the summaries, the paper doesn't mention if it's used as augmentation of the dataset or if it's a bug - removed. Player initials "C.J. McCollum" in table "CJ McCollum" in text. 

\subsection{Transformations of player names, city names and \linebreak[4] team names}
The motivation - making data denser. In a lot of summaries the player is firstly introduced "It was up to LeBron James to take over for Cleveland" and then referenced only by his surname "James finished with 27 points, 14 assists and 8 rebounds..." Many transformations were ommitted although possible. The text looks more natural if the network learns that Philladelphia is sometimes mentioned as Philly. Those nuances are left in the text.

\subsection{Other transformations made to the dataset}
Here I'll mention lowercasing, number transformations.

\subsection{Byte pair encoding}
What it is, who introduced it. How it decreases the number of tokens.

\subsection{The statistics of the transformed dataset}
What is achieved with the use of all the mentioned transformations. The number of unique tokens decreased almost than 4 times (11 300 to 2 900). The fraction of tokens mentioned more than 5 times increased from 47\% to 89,5\%. Highlight the differences to processing of  